\ El segundo trabajo práctico de Organización del Computador II tiene por objetivos tanto ejercitar el uso de instrucciones SIMD, 
como desarrollar un informe riguroso que presente comparaciones de distintas implementaciones de un mismo algoritmo y experimentos en base 
a los distinos filtros especificados en la consigna. 
\par SIMD (Single Instruction, Multiple Data) es un modelo de procesamiento de datos vectorial, cuya principal diferencia con las instrucciones escalares
tradicionales (por ejemplo, las que trabajan con los registros de propósito general en una arquitectura x86) es que las primeras son capaces de operar sobre varios
datos en paralelo (por ejemplo, sumar 8 pares de números y guardar el resultado en un registro).
\par En 1999 Intel introdujo a su línea de CPUs 70 nuevas instrucciones como parte del estandar SSE (Streaming SIMD Extensions), agregándole capacidad de procesamiento
vectorial a la arquitectura x86. La primer parte de este trabajo práctico consiste en implementar filtros para imágenes utilizando dichas instrucciones, de modo que se 
procesen al menos 2 píxeles simultáneamente por cada iteración de los algoritmos presentados.
\par En este informe se busca evidenciar las diferencias de rendimiento que hay entre una implementación de un filtro escrita en lenguaje ensamblador utilizando instrucciones SIMD, 
y una implementación en lenguaje C del mismo donde sólo se procesa un píxel por cada iteración del ciclo. Se espera observar un rendimiento superior en la implementación 
en lenguaje ensamblador, y una brecha que se acorta cada vez más a medida que aumentamos el nivel de optimización en el compilador del lenguaje C.
\par Finalmente, los experimentos realizados buscarán proveer un entendimiento más profundo en cuanto al funcionamiento de los filtros y de la arquitectura x86. En primer lugar, 
se evaluará cuánta precisión se pierde y cuánto rendimiento se gana al reemplazar el uso de operaciones en punto flotante por distintas aproximaciones con enteros en el filtro Imagen Fantasma. 
El último experimento explorará el comportamiento de la CPU frente a distintos escenarios relacionados con los saltos condicionales. 