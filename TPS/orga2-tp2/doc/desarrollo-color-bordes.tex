\subsection{Función Color Bordes}

\par El objetivo de este filtro es asignar a cada píxel que forma parte de un borde en la imagen,
un valor calculado en base a los colores de los píxeles que lo rodean.

\par Esto sucede mediante un cálculo que toma el valor absoluto de la diferencia entre los
valores BGR (blue, green y red ) de cada píxel, haciendo esto sobre un eje vertical y
horizontal. Es decir, este cálculo se aplica a aquellos valores que rodean a cada píxel
por encima, por debajo (eje y) y por los costados (eje x), y de ese cálculo sale un
único valor que se le asigna al píxel de salida.
\par Nótese también que los píxeles tienen una componente de transparencia $a$, siendo la
codificacion de los píxeles de la forma BGRA (blue, green, red y alpha), pero en este caso
no se opera sobre la componente de transparencia, solo aquellas de colores.

\par Ahora que fue presentada la idea para la comprensión general del funcionamiento del filtro,
se procede a una descripción detallada de una iteración dentro de la función que
la implementa:



\par Como el filtro requiere que los píxeles de los bordes no sean iterados, le restamos al alto original el valor 2, ya que
respecto a la altura, la primer y última fila de píxeles no van a ser tomadas en cuenta.
En el caso del ancho, lo que se hace es tomar dos registros, donde uno va a ser el encargado
de guardar el ancho original de la matriz, y el segundo va a tener el ancho sobre el
que efectivamente se va a operar. Este último será dividido por 2 (porque se procesarán 2 píxeles en cada iteración) y luego restado uno a este
último resultado, ya que hay 2 píxeles que no van a ser iterados, que son los de los bordes izquierdo y derecho.
\\
\\
$\hspace*{2cm}$Representación de las posiciones sobre las que no se aplica el filtro
			\\
			\\
		$\hspace*{4cm}$$|$ X $|$ \ \ X  \ \ $|$ \ \ X \ \  $|$ \ \ X \ \  $|$ \ \ X \ \  $|$ \ \ X \ \  $|$ \ \ X \ \  $|$ \ \ X \  $|$ X $|$$\\$
		$\hspace*{4cm}$$|$ X $|$ p11 $|$ p12 $|$ p13 $|$ p14 $|$ p15 $|$ p16 $|$ p17 $|$ X $|$$\\$
		$\hspace*{4cm}$$|$ X $|$ p21 $|$ p22 $|$ p23 $|$ p24 $|$ p25 $|$ p26 $|$ p27 $|$ X $|$$\\$
		$\hspace*{4cm}$$|$ X $|$ p31 $|$ p32 $|$ p33 $|$ p34 $|$ p35 $|$ p36 $|$ p37 $|$ X $|$$\\$
		$\hspace*{4cm}$$|$ X $|$ \ \ X \ \  $|$ \ \ X \ \  $|$ \ \ X \ \  $|$ \ \ X \ \  $|$ \ \ X \ \  $|$ \ \ X \ \  $|$ \ \ X \  $|$ X $|$
				\\
				\\
				\par Luego de toda esta operatoria, pero antes de aplicar el ciclo del filtro, se deben poner en blanco la fila superior de la imagen
				destino, lo cual se logra con un pequeño loop
				donde se toma una máscara (una secuencia de 128 bits seteados en 1, definida en la sección
				rodata) y es utilizada para iterar sobre todos los píxeles de la primer fila y setear sus valores en 255, es decir, color blanco.
				Es posible iterar de a 4 píxeles ya que se tiene la garantía
				(por lo indicado por la cátedra) que el ancho de la matriz es máltiplo de 4, por lo que
				siempre entrarán los 16 bytes que se insertan en esas posiciones de la matriz.


				\par Hecho esto, se actualizan las direcciones a las que apuntan los registros de origen y destino para que pasen a apuntar
				a las posiciones (1, 1) de sus respectivas matrices y se carga en xmm14 el valor de una máscara que será utilizada para mantener el valor inicial de las
				componentes de transparencia de los diferentes píxeles.
				\\
				\\
			$\hspace*{5.5cm}$$|$ X $|$ \ \ X  \ \ $|$ \ \ X \ \  $|$ \ \ X \ \  $|$ \ \ X \ \  $|$...$\\$
		$\hspace*{5.5cm}$$|$ X $|$ p11 $|$ p12 $|$ p13 $|$ p14 $|$...$\\$
				$\hspace*{6.4cm}$$\uparrow$
							\\
							\par En este momento se ingresa por primera vez al ciclo de filas ($rowLoop$), que es el encargado
							de
							\begin{itemize}
								\item resetear el contador de las columnas por las que se itera
								\item de controlar que se opere
								      sobre todas las filas necesarias, sin excederse ni acceder a memoria que no le pertenece al programa
								\item setear el primer y último píxel en blanco
							\end{itemize}.

							\par Para llevar a cabo esto, se guarda en un registro el contador de columnas a recorrer, se setea el primer píxel en blanco, y se ingresa al
							loop de columnas, que va a ser el encargado de realizar las operaciones de diferencia
							vertical y horizontal sobre los píxeles de la matriz.

							\par Es necesario usar un registro xmm para acumular los resultados de estas operaciones, por lo que éste se inicializa con el valor del
							registro xmm14 (la máscara para los valores alpha de cada píxel), para asegurar que la componente $a$ de cada píxel tendrá el valor 255, tal como pide la consigna.

							\par Se decidió operar sobre 2 píxeles en simultáneo ya que, al descartar el primer y último píxel de cada fila, ya no se cuenta con la certeza de que
							la cantidad de píxeles a procesar por fila es múltiplo de 4, por lo que no es posible ir avanzando y operando de a 4 píxeles por iteración.

							\par Luego se levantan los valores de cada una de las componentes de los 2 píxeles (bgra) extendidas de 8 bits a 16 para evitar
							una posible perdida de precisión, y se utilizan 2 registros xmm para almacenar los valores de 2 pares de píxeles: en el
							primero, los dos píxeles superiores izquierdos a donde se pretende aplicar el filtro; y en el segundo los inferiores izquierdos.
							\\
							\\
						$\hspace*{5.5cm}$$|$ p00 $|$ p01 $|$ p02 $|$ p03 $|$ p04 $|$...$\\$
				$\hspace*{5.5cm}$$|$ p10 $|$ p11 $|$ p12 $|$ p13 $|$ p14 $|$...$\\$
		$\hspace*{6.7cm}$$\uparrow$$\hspace*{0.8cm}$$\uparrow$$\\$$\\$
$\hspace*{5.5cm}$$\Rightarrow$ xmm0$=$$|$ p00 $|$ p01 $|$$\\$
$\hspace*{5.5cm}$$\Rightarrow$ xmm1$=$$|$ p20 $|$ p21 $|$$\\$

\par A estos valores, se les aplica la resta componente a componente (los 16 bits de la
componente b del primero contra los de la componente b del segundo, y lo mismo con los demás) y luego, a cada uno de estos valores
resultantes, se le toma el valor absoluto. De esta manera, tenemos en el primer registro el resultado del cálculo:\\
$\hspace*{2.5cm}$xmm0 $=|\ $$abs(p00(bgra) - p20(bgra))$$\ | \ $$abs(p01(bgra) - p21(bgra))$$\ |$

	\par Estos valores son sumados al registro acumulador definido anteriormente.
	Notar que al hacer esto ocurren 2 cosas: primero, el resultado del cálculo está
	dividido para los dos píxeles que que se pretenden procesar, es decir, fue posible el cálculo
	necesario para 2 píxeles en solamente una aplicación de las instrucciones. Y segundo,
	se mantienen los valores de la componente alpha, porque la resta entre las componentes alpha
	de los píxeles procesados da 0 ya que son todas iguales, por lo que mantiene el valor que venía del primer seteo sobre el acumulador.

	\par En la próxima serie de instrucciones (2 repeticiones más de la diferencia vertical), se ejecuta el
	mismo código pero cambiando los pares de píxeles que se guardan en los registros xmm:
	los píxeles directamente superiores e inferiores a los píxeles (1, 1) y (1, 2) y por último los píxeles superiores e inferiores
	derechos en cada uno de los bloques de instrucciones respectivos.
	\\
	\\
$\hspace*{1.5cm}$$|$ p00 $|$ p01 $|$ p02 $|$ p03 $|$ p04 $|$...$\\$
$\hspace*{1.5cm}$$|$ p10 $|$ p11 $|$ p12 $|$ p13 $|$ p14 $|$...$\\$
$\hspace*{2.7cm}$$\uparrow$$\hspace*{0.8cm}$$\uparrow$$\\$$\\$
En el segundo bloque de código:\\
$\hspace*{1.5cm}$$\Rightarrow$ xmm0$=$$|$ p01 $|$ p02 $|$$\\$
$\hspace*{1.5cm}$$\Rightarrow$ xmm1$=$$|$ p21 $|$ p22 $|$$\\$
En el tercer bloque de código:\\
$\hspace*{1.5cm}$$\Rightarrow$ xmm0$=$$|$ p02 $|$ p03 $|$$\\$
$\hspace*{1.5cm}$$\Rightarrow$ xmm1$=$$|$ p22 $|$ p23 $|$$\\$

\par Una vez terminada la ejecución de la diferencia vertical de esta iteración, se comienza con la diferencia horizontal. A diferencia del 
cálculo vertical, en este caso las direcciones de memoria a las que se deben acceder por cada bloque de código pertenecen a la misma fila.
Siguiendo con el ejemplo del filtro para los píxeles (1, 1) y (1, 2), los registros
por cada bloque de código quedan de la siguiente manera:
\\
\\
		$\hspace*{1.5cm}$$|$ p00 $|$ p01 $|$ p02 $|$ p03 $|$ p04 $|$...$\\$
				$\hspace*{1.5cm}$$|$ p10 $|$ p11 $|$ p12 $|$ p13 $|$ p14 $|$...$\\$
						$\hspace*{2.7cm}$$\uparrow$$\hspace*{0.8cm}$$\uparrow$$\\$
							En el primer bloque de código:\\
						$\hspace*{1.5cm}$$\Rightarrow$ xmm0$=$$|$ p00 $|$ p01 $|$$\\$
				$\hspace*{1.5cm}$$\Rightarrow$ xmm1$=$$|$ p02 $|$ p03 $|$$\\$
			En el segundo bloque de código:\\
		$\hspace*{1.5cm}$$\Rightarrow$ xmm0$=$$|$ p10 $|$ p11 $|$$\\$
$\hspace*{1.5cm}$$\Rightarrow$ xmm1$=$$|$ p12 $|$ p13 $|$$\\$
	En el tercer bloque de código:\\
	$\hspace*{1.5cm}$$\Rightarrow$ xmm0$=$$|$ p20 $|$ p21 $|$$\\$
			$\hspace*{1.5cm}$$\Rightarrow$ xmm1$=$$|$ p22 $|$ p23 $|$$\\$
\\
Y respectivamente, las operaciones quedan:\\
En el primer bloque de código:
$abs(|px00|px01| - |px02|px03|)$\\
En el segundo bloque de código:
$abs(|px10|px11| - |px12|px13|)$\\
En el tercer bloque de código:
$abs(|px20|px21| - |px22|px23|)$
\\
\\
\par Nótese también que en el caso anterior no se tomaba el valor de los píxeles (1, 1) y (1, 2)
para el cálculo de la diferencia, pero en este caso es necesario tomar el de (1, 1) para
poder aplicar el filtro al (1, 2). Las instrucciones utilizadas son exactamente las mismas que se usaron para la diferencia
vertical, así como las sumas al registro acumulador.

\par De esta manera, ya quedan en xmm8 (registro acumulador) los valores correspondientes
a cada uno de los píxeles a los que se debía aplicar el filtro, pero aún hay 2 problemas:
\begin{itemize}
	\item los valores de cada componente de los píxeles estan expresados en 16 bits, cuando los píxeles almacenan
			valores de 8 bits
	\item estos valores pueden haber excedido la representación no signada de los valores de colores (el rango de valores que
			necesitamos para representar los valores es un rango no signado de numeros entre el 0 y
			el 255 en representación decimal, pero pueden haberse excedido de esta representación
			luego de las diferentes operaciones aplicadas)
\end{itemize}

\par Para resolver estos problemas, se utilizó una instrucción de empaquetado provista por el set de instrucciones de SSE, que trata a ambos: 
empaqueta valores de 16 bits a en valores de 8 bits, y además nos devuelve una saturación no signada (si fuera signada, el rango de saturación no sería
de 0 a 255 sino de -128 a 127, que no es lo que queremos para representar colores).
\par Una vez empaquetados los valores, quedan almacenados en la parte baja de mi registro
acumulador, por lo que se usa una operación para mover qwords (son 8 valores de 8 bits) y ponerlo
en su posicion dentro de la matriz de destino.
\\
$\hspace*{5cm}$$Matriz \ Destino \ actual $$\\$
$\hspace*{3.5cm}$$|$ 255 $|$ \ \ \ \ \ \ \ 255 \ \ \ \ \ \ \ $|$ \ \ \ \ \ \ \ 255 \ \ \ \ \ \ \ $|$ 255 $|$ 255 $|$...$\\$
$\hspace*{3.5cm}$$|$ 255 $|$ filtered(p11) $|$ filtered(p12) $|$ \ \ ? \ \ $|$ \ \ ? \ \ $|$...$\\$
$\hspace*{5.5cm}$$\uparrow$$\hspace*{2cm}$$\uparrow$

\par A esta altura, los píxeles (1, 1) y (1, 2) ya se encuentran con los valores del filtro aplicados, pero
todavia falta terminar la iteración para que lo mismo funcione para el resto de píxeles.
Entonces, se avanzan los punteros usados para acceder a las posiciones de memoria de las
matrices de origen y destino en 8 bytes, para que tomen el próximo par de columnas.
Luego de esto, se decrementa el contador de columnas y se efectúa un salto condicional (porque la
operación de decremento ya actualiza los flags acorde a su resultado) para decidir si
se vuelve a ciclar sobre las columnas, o habría que pasar a la próxima fila.

\par En este caso corresponde volver a ejecutar el ciclo de las columnas, corrido a los 2
píxeles que siguen, pero se explicarán 2 casos mas:
\begin{itemize}
\item el caso donde hay que avanzar a la próxima fila
\item el caso donde ya se terminaron las filas y las columnas
\end{itemize}
$\rightarrow$ En el primero de estos, el contador de columnas habrá quedado en 0,
por lo que pasará a setear el último píxel de la fila (el píxel de la última columna de
la fila sobre la que estaba iterando) en blanco. También avanzaría los punteros hasta la
columna 1 de la fila siguiente para dejarlos preparados para la próxima ejecución
del ciclo de filas. Finalmente, decrementaría el contador de filas, y si aún quedan filas
por recorrer, saltaría al ciclo de filas, reseteando el contador de columnas y poniendo el
primer píxel de la nueva fila en blanco.\\
$\rightarrow$ En el caso donde ya se recorrió la última fila entera, ocurrirá lo que fue comentado
antes, pero en vez de saltar de nuevo al ciclo de las filas, pasaría a ejecutar un último
ciclo donde se setea la última fila de la matriz destino en blanco, tal y como lo hizo antes
de empezar a ejecutar el primer ciclo de filas.
\newpage
Representación de la matrix destino finalizada
\\
\\
$\hspace*{1.5cm}$$|$ 255 $|$ \ \ 255 \ \ $|$ \ \ 255 \ \ $|$ \ \ 255 \ \ $|$ \ \ 255 \ \ $|$ \ \ 255 \ \ $|$ \ \ 255 \ \ $|$ \ \ 255 \ \ $|$ 255 $|$$\\$
$\hspace*{1.5cm}$$|$ 255 $|$ f(p11) $|$ f(p12) $|$ f(p13) $|$ f(p14) $|$ f(p15) $|$ f(p16) $|$ f(p17) $|$ 255 $|$$\\$
$\hspace*{1.5cm}$$|$ 255 $|$ f(p21) $|$ f(p22) $|$ f(p23) $|$ f(p24) $|$ f(p25) $|$ f(p26) $|$ f(p27) $|$ 255 $|$$\\$
$\hspace*{1.5cm}$$|$ 255 $|$ f(p31) $|$ f(p32) $|$ f(p33) $|$ f(p34) $|$ f(p35) $|$ f(p36) $|$ f(p37) $|$ 255 $|$$\\$
$\hspace*{1.5cm}$$|$ 255 $|$ \ \ 255 \ \ $|$ \ \ 255 \ \ $|$ \ \ 255 \ \ $|$ \ \ 255 \ \ $|$ \ \ 255 \ \ $|$ \ \ 255 \ \ $|$ \ \ 255 \ \ $|$ 255 $|$
\\

\par Es evidente que la técnica utilizada para desarrollar este filtro fue la de $operatoria \
con \ kernel$, tomando 2 kernels (que no fueron aplicados estrictamente como se vio en la clase),
pero que tienen la siguiente forma:$\\$
$\hspace*{1.5cm}$__________$\hspace*{3cm}$$\hspace*{1.7cm}$__________$\\$
Kernel 1: $|$ 1$|$ 1$|$ 1$|$ $\hspace*{3cm}$Kernel 2: $|$ 1$|$ 0$|$ -1$|$ $\\$
$\hspace*{1.5cm}$$|$ 0$|$ 0$|$ 0$|$ $\hspace*{3cm}$$\hspace*{1.47cm}$$|$ 1$|$ 0$|$ -1$|$ $\\$
$\hspace*{1.5cm}$$|$-1$|$-1$|$-1$|$ $\hspace*{3cm}$$\hspace*{1.43cm}$$|$ 1$|$ 0$|$ -1$|$ $\\$
$\hspace*{1.5cm}$__________$\hspace*{3cm}$$\hspace*{1.7cm}$__________$\\$