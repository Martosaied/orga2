Este es un archivo escrito en código de máquina, aquí definimos las rutinas de atención a interrupciones que son indicadas desde los descriptores de la IDT
Así como también definimos las rutinas de atención a excepciones (que también se definen en la IDT).
\\\\
El archivo se divide en 3 partes distintas donde se definen e implementan distintas rutinas.
Primero vemos la definición de 2 macros, \emph{ISR_CON_ERROR_CODE e ISR} las cuales definen la rutina de atención a las excepciones.
La primera es utilizada para definir la rutina a excepciones que devuelven un errorCode en la pila a la hora de cambiar el stack por uno de privilegio 0.
La segunda es utilizada para definir la rutina de excpeción que no utiliza ningún tipo de errorCode, por lo que la estructura de la pila que queda es distinta.
\\\\
La única diferencia entre ambas macros es la primera instrucción donde en la utilizada para excepciones con errorCode pasamos el valor ubicado en la posición a la que apunta ESP que, como la documentación de Intel 
indica, es donde se encuentra el errorCode.
\\
El resto de la implementación de las macros es exactamente igual. Basado en la variable global de modo_debug decidimos si imprimir o no la excepción en pantalla pasando por la pila como párametro
todos los registros e información a mostrar. De no tener el modo debug activo salteamos este call a imprimir_excepción y pasamos a desalojar la tarea que generó la excepción.
En la etiqueta error_enable se encuentra un byte que indica si hubo una excepción y luego utilizar esta información para parar la ejecución del scheduler.
\\\\
En la segunda parte del archivo vemos las rutinas de atención a las interrupciones de teclado y clock.
La rutina de interrupción de reloj primero chequea si error_enable esta en 1, como dijimos anteriormente. En caso de estarlo, detiene cualquier tipo de switch de tarea.
Luego tenemos dos calls, uno para actualizar el reloj de abajo a la derecha y otro para actualizar el reloj que se encuentra debajo del índice de la tarea que se está ejecutando para
ser capaces de ver el funcionamiento y el cambio de tarea sin necesidad de utilizar breakpoints.
\\
Después de esto chequeamos si el juego finalizó, y de ser así, imprimimos un mensaje en pantalla indicando el resultado del juego y dejamos el juego en un while infinito.
\\
Luego, en caso de que el juego no haya terminado aún, buscamos qué tarea es la próxima a ejecutarse utilizando sched_next_task, verificamos que la tarea devuelta no sea la actual. En caso de cumplirse las
condiciones, cargamos el selector de segmento que apunta al descriptor de TSS dentro de la GDT para de esta forma producir el task switch.
\\
La rutina de interrupción de teclado solo se preocupa por una cosa: una vez cargado en AL lo que se encuentro en el puerto 0x60, chequeamos si lo que fue apretado fue una "Y".
En caso de haber presionada dicha tecla y estar en modo_debug, ejecutamos restaurar pantalla (esta función solo restaura la pantalla en caso de que haya habido una excepción, sino no hará nada)
y desactivamos el error_enable.
\\
Si no se encontraba el modo_debug activo, lo activaremos.
\\\\
Por último, en el archivo tenemos las rutinas de atención a las syscalls que pueden producir las diferentes tareas. No nos detendremos a explicar qué hace cada una ya que éstas
se encuentran implementadas en game.c y aquí solo nos encargamos de atender la interrupción correctamenete, preservar todos los registros y pasar los parametros a la función de C que
se encargara de ejecutar la lógica.
\\
Como añadido, en este archivo decidimos implementar el backtrace, el cual se encuentra en el final del mismo. Esta rutina toma dos parámetros: el primero es un puntero a un arreglo donde se deben guardar 
las direcciones que vamos encontrando. El segundo parámetro es el valor que tenía el registro ebp cuando se empezó a ejecutar el handler de excepción. El algoritmo va buscando el base pointer anterior a donde
está parado y también obtiene la return address asociada a ese stackframe y la guarda en el arreglo de salida. Esto se repite un máximo de 5 veces, pero corta si el base pointer es nulo, no es múltiplo de 4, o 
si está fuera del rango del código de los jugadores.