En este archivo se encuentran las definiciones iniciales de las TSS de las tareas INIT, IDLE, Rick y Morty. Así como también está definido
de manera global 2 arrays en los que se encuentran las TSS de los Mr. Meeseeks de Morty y otro para los Mr. Meeseeks de Rick. Este último array tiene una correspondencia directa con los
arrays de tareas definidas en el Scheduler. Donde el índice en el array del schduler corresponde al mismo índice - 1 en los arrays de TSS. \\
La definición de las primeras 4 tareas mencionadas ocurre en tiempo de compilación para casi toda sus campos.
 
Luego en \emph{tss_init}, la cual es ejecutada en el kernel, nos encargamos de cargar los valores que son obtenidos dinámicamente y de completar todas las TSS de los Mr. Meeseeks.
\\
Dentro de los campos que completamos de esta manera están:
   \begin{itemize}
       \item El CR3 el cual es obtenido a partir de inicializar el mapa de memoria de las tareas Rick y Morty. Luego dichos CR3 son utilizados en las tareas meeseeks también ya que comparten mapas de memorias.
       \item La pila de nivel de 0 para todas las tareas, para la cual pedimos una página libre del kernel.
   \end{itemize}
Algo para aclarar, debido a que las TSS de cada tarea se encuentran definidas en este archivo, decidimos completar la base de los índices de la GDT de las tareas en tss_init.
Obteniendo la GDT es fácil modificar esta información, ya que los índices de los meeseeks corresponden a sus indice en la GDT + un índice base (el índice en la gdt) de Rick o Morty respectivamente. 
El resto de los campos, como dijimos, en su mayoría fueron completados a partir de la consigna.
\\
Una última aclaración, en tss.c se puede cómo creamos las tss de los Mr Meeseeks con EIP y ESP en 0.
Esto se debe a que luego, en game.c a la hora de efectivamente crear una nueva tarea Mr. Meeseeks seteamos el EIP y el ESP mapeando el código que ejecuta la tarea
desde el área virtual de los Mr. Meeseeks (organizado a partir de sus índices) a la celda en la que se encontrara en el mapa. Daremos más detalles
a la hora de explicar game.c.
