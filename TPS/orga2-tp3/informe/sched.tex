En este archivo primero que nada, nos definimos 2 arreglos (inicializados en 0) de 11 elementos cada uno,
llamados \emph{tasks_rick} y \emph{tasks_morty}, los cuales se encargarán de almacenar estructuras de datos
con información necesaria para todas las tareas de Rick y de Morty respectivamente. 
Es decir, almacenarán structs de las tareas principales de cada uno de los jugadores y
sus 10 posibles \emph{"subtareas"} MrMeeseeks.
\\
Estas estructuras están declaradas en el archivo sched.h y proveen datos sobre:
\begin{itemize}
    \item idx_gdt: el índice del descriptor de la tarea en la gdt.
    \item es_jugador: si este bit está encendido, la tarea es Rick o Morty, si esta apagado 
    es un MrMeeseeks de alguna de las tareas anteriores.
    \item y,x: posiciones de la tarea sobre los ejes del mapa del juego.
    \item cap_movimiento: capacidad de movimiento de la tarea (válido en caso de ser MrMeeseeks).
    \item uso_portal_gun: indica si la tarea ya utilizó su portal gun o no.
    \item status: indica si la tarea es parte de las tareas activas para que el scheduler
    sepa si debe alocarla para que sea ejecutada o no. De no serlo, la saltea.
    \item isrNumber: atributo utilizado para el display del reloj de cada tarea.
\end{itemize}

Al inicializar el scheduler, seteamos los valores correspondientes a las tareas de
Rick y Morty, con el bit de status en 1 para ambos, y asignándole a cada uno de los 
elementos del arreglo su índice correspondiente en la GDT (junto con su nivel de privilegio).

Creamos tambien ciertas variables globales que nos van a ayudar a conocer el estado de las tareas,
la tarea actual y si el juego finalizó.

Adentrándonos más en la funcionalidad del scheduler en sí, tenemos una serie de funciones
que son útiles para distintas ocasiones en el juego:
\begin{itemize}
    \item _maybe_degradar_cap: simplemente se encarga de disminuir la capacidad de movimiento de la tarea
    actual, utilizando el atributo antes mencionado de \emph{cap_movimiento}, que tal como lo indica el enunciado
    se disminuye cada 2 clocks sobre la tarea indicada. En el caso de que la tarea sea Rick o Morty y no
    uno de sus MrMeeseeks no tiene efecto.
    \item sched_next_task: ésta función es muy importante porque es la encargada de devolverme
    el índice de la gdt de la siguiente tarea en la lista de tareas del scheduler (formato round-robin).
    Aquí primero chequeamos si la tarea anterior a la que debemos devolver fue una de Rick
o una de Morty, y hacemos un \emph{"toggle"} de la variable global \emph{ultimo_fue_rick} segun corresponda, y luego
(tambien dependiendo del caso anterior para determinar el caso siguiente) buscamos la siguiente tarea con
status=1 para saber cual es la que corresponde devolver para hacer el switch. Notamos también que 
como estamos en formato Round-Robin, si se llega al final de la lista de tareas, se vuelve a la primer
posición del arreglo.
    \item reset_esp: reseteamos el esp0 de la tarea corriéndolo hasta la base de su pila.
    \item desalojar_tarea: usamos ésta función para desalojar la tarea actual, seteando su
status en 0 (ie, sacándola de la lista de tareas disponibles para alocar con el scheduler), reseteando
el esp con la funcion explicada arriba y usando el cr3 actual para desmapear el area de memoria
de la tarea, haciendo el cálculo de la direccion base de la memoria virtual, sumado a 2 veces el 
tamaño de las páginas (ya que sabemos que cada tarea ocupa 2 páginas) y usamos el índice de la tarea
para terminar de ubicar la dirección de memoria a desmapear. En caso de que la tarea a desalojar
sea la de Rick o Morty, directamente devolvemos el selector de la tarea IDLE.
    \item desalojar_tarea_indice: se utiliza únicamente para el caso de \emph{use_portal_gun}, desalojando una tarea
cuyo índice es pasado por parametro. Se desaloja de la misma manera que se desalojaba en la funcion
\emph{desalojar_tarea}, pero en este caso siendo una funcion void y usando el índice recibido para ubicar la
tarea dentro de los arreglos de tasks mencionados anteriormente.
\end{itemize}