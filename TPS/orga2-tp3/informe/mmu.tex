El archivo de la MMU es uno corto pero que lleva el peso de la asignación y mapeo de páginas para
tareas, por lo que es esencial para el correcto funcionamiento de la distribución de memoria.
Para la inicialización de la MMU lo único que hacemos es asignar a la variable global
que definimos como \emph{proxima_pagina_libre} el puntero a la primer dirección de
páginas libres, para poder usarlo después.
Siguiendo con una descripción de las diferentes funciones definidas en este archivo, explicamos:
\begin{itemize}
    \item mmu_next_free_kernel_page: simplemente la utilizamos para conseguir la próxima página libre
    del kernel, tal como lo indica su nombre. Ésto lo hacemos devolviendo el valor de la variable de
    \emph{proxima_pagina_libre} y actualizando su valor a la siguiente página (sumándole el tamaño de una página).
    \item mmu_init_kernel_dir: es la encargada de realizar el mapeo con \emph{identity mapping} de los primeros 4Mb
    de memoria que son dedicados al kernel. Esto lo hacemos agarrando un puntero a la dirección del Page Directory y
    de la Page Table dados por la cátedra en el enunciado. Inicializamos en 0 las 1024 entradas de cada una
    y seteamos la base y los atributos necesarios en la primer entrada del PD, siendo que va a apuntar al PT que 
    mencionamos arriba. También, seteamos los atributos de todas las entradas de la PT. Notar que como estamos haciendo
    \emph{identity mapping}, armamos un ciclo que hace corresponder la direccion fisica asociada como base, con el índice
    de cada entrada en la PT. Finalmente, retornamos el puntero a la PD.
    \item mmu_map_page: esta función hace el mapeo de una dirección virtual a una dirección física, utilizando el CR3 para
    determinar la dirección de la PD, que los recibimos por parámetro. Ésto lo hacemos primero obteniendo los bits de la dirección
    de memoria virtual que representan la dirección del PD, y obteniendo también el índice a la PT correspondiente.
    Con el CR3 recibido por parámetro y el \emph{directoryIdx}, logramos llegar al descriptor de tabla de páginas, chequeamos su bit
    de present, que si está en cero, nos va a indicar que tenemos que ir a buscar una nueva página para asignarle una nueva PT, la cual
    inicializamos con todas sus entradas en 0, y al descriptor de tabla de páginas le seteamos el bit present en 1. Nuevamente, asignamos los atributos
    necesarios. Notar que no nos preocupamos por los niveles de privilegio porque confiamos en que los parámetros que ingresan son correctos.
    Finalmente, accedemos a la entrada de la tabla de páginas, y asignamos a la entrada correspondiente, dada por el table index que obtuvimos
    antes, la dirección física pasada por parámetro, junto con sus atributos. Para terminar con el mapeo, hacemos un tlbflush (proporcionado por la cátedra).
    \item mmu_unmap_page: seguimos la misma lógica que el map, pero seteamos el bit de present de la PTE en 0.
    \item mmu_init_task_dir: la utilizamos para crear el esquema de paginación de una tarea. Lo que hacemos es tomar una nueva página que
    representa un nuevo Page Directory, cuyas entradas inicializamos todas en 0, y al igual que en \emph{mmu_map_page}, seteamos la primer entrada
    con sus atributos, apuntando además a la tabla de páginas del kernel. Utilizamos \emph{mmu_map_page} para mapear el codigo de la tarea en el directorio de la tarea,
    y también hacemos \emph{identity mapping} para el área de kernel. Copiamos la memoria de codigo desde la dirección de
    \emph{code_start} que recibimos por parámetro, a la dirección física apuntada por la virtual que obtuvimos antes para realizar el \emph{identity mapping}, y finalmente
    desmapeamos éste area que ya no necesitamos mantener mapeada. Para terminar, usamos un tlbflush al igual que en las funciones anteriores. 
\end{itemize}