En este archivo decidimos poner toda la lógica de funcionamiento del juego y de los servicios expuestos a las tareas. También nos vamos a encontrar con 3 variables locales: un arreglo de tamaño 40 para almacenar las posiciones de las 40 megasemillas y si fueron absorbidas, y dos variables numéricas que llevan cuenta del puntaje de cada jugador.
\\\\
La función _move_meeseeks toma un directorio de páginas, un par de coordenadas x, y de origen y un par de coordenadas x, y de destino, para mover el código y pila de un Mr. Meeseeks a su nueva posición. Para lograr esto, nosotros pensamos la porción de la memoria correspondiente al mapa como una matriz de 80x40. Usando las coordenadas viejas y nuevas calculamos la posición en la memoria física que corresponde a cada celda en cuestión. Luego, habiendo calculado ambas direcciones físicas, mapeamos temporalmente 2 páginas a partir de cada dirección con identity mapping en el cr3 pasado como parámetro (esto no es un problema ya que el espacio virtual del 0x4000000 al 0x1CFFFFF no está mapeado a priori por ningún directorio) para luego copiar todo el contenido de la celda vieja a la nueva (esto es: código y pila de nivel 3) y dejar la celda vieja llena de ceros. Hecho esto desmapeamos las cuatro páginas mencionadas anteriormente. Finalmente, buscamos la tss del Mr. Meeseeks que deseamos mover y su índice dentro de su arreglo de 10 tss’s de Mr. Meeseeks. Dado el índice del 0 al 9 en su correspondiente arreglo de tss’s, nosotros decidimos calcular la dirección virtual de un Mr. Meeseeks dado como MEESEEKS_VIRT_START + 2 * PAGE_SIZE * índice.
Llamamos mem_slot a esta variable, y usando el cr3 de la tss del Mr. Meeseeks desmapeamos las 2 páginas a partir de la dirección virtual mem_slot y luego la mapeamos usando como dirección física la nueva celda dentro del mapa donde estará la tarea.
\\\\
En la función game_checkEndOfGame primero recorremos todas las semillas para ver si queda alguna presente, y nos fijamos si la tarea Rick y/o la tarea Morty fueron desalojadas. En caso de que alguna de estas condiciones se cumpla, imprimimos en la pantalla la información correspondiente y nos quedamos ciclando indefinidamente. Como esta función se ejecuta antes de sched_next_task(), en caso de que el juego termine siempre se quedará en el while y cuando haya una interrupción de reloj entraremos a esta función de vuelta antes de tener la oportunidad de cambiar de tarea, por lo que el juego quedará tildado en la pantalla de finalización.
\\\\
La función move contiene toda la lógica necesaria para atender a la interrupción que lleva el mismo nombre. Primero checkea que no la hayan llamado ni Rick ni Morty. De ser así termina y desaloja la tarea actual (la que la llamo: Rick o Morty). En caso contrario calcula cuánta distancia manhattan pretende moverse el Mr. Meeseeks. Si es mayor a la que tiene permitido, el servicio termina y devuelve 0. La razón por la cual le tomamos el valor absoluto al campo de capacidad de movimiento es porque decidimos (para evitar usar más campos en nuestros structs y más variables globales, que bastantes tiene el TP) que el valor, por cada vez que la tarea sea ejecutada, se disminuya de la siguiente manera: 7, -7, 6, -6, 5, -5, … hasta llegar a 1. Luego de este checkeo, borra la celda actual del Mr. Meeseks y actualiza los campos x e y de su task con las nuevas coordenadas. Si hay una semilla en el campo actual, la absorbe y la tarea es desalojada. Caso contrario, imprime una “k” en la nueva celda y llama a _move_meeseeks para mover el código y datos de su tarea a la nueva posición que corresponda.
\\\\
La función create_mrmeeseeks toma un puntero al código de la función que controla al nuevo Mr. Meeseeks, y un par de coordenadas x, y para ubicarlo. Si el puntero al código esta fuera del rango virtual 0x1D00000, 0x1D03FFF, entonces la llamada al servicio es inválida y la tarea que la llamó sera desalojada. También sucederá esto si llama a este servicio un Mr. Meeseeks. Si las coordenadas están fuera del rango de la pantalla, el servicio termina y devuelve 0. Acto seguido, buscamos en el arreglo correspondiente de tasks (depende si llamó Rick o Morty al servicio) la primer tarea Mr. Meeseeks que tenga status 0. Si no encontramos ninguna disponible el servicio devuelve 0, y sino nos quedamos con el índice, al cual llamaremos i. Luego checkeamos si en la posición pasada había una semilla, y si la había procedemos tal como hicimos en move. Luego de estos checkeos, procedemos a crear el nuevo Mr. Meeseeks. Calculamos la posición de la memoria física correspondiente a la posición x, y, mapeamos dos páginas a partir de esa posición con identity mapping, y copiamos en la primer página el código a partir del puntero y seteamos la segunda página (donde estará la pila) con ceros. Hecho esto desmapeamos la dirección de la celda y hacemos el mismo procedimiento de mem_slot que describimos en move. Por último, seteamos el eip de la tss correspondiente en la primera posición de las direcciones virtuales asignadas a este Mr. Meeseeks (más conocida como mem_slot), seteamos el esp en el final de la segunda de las páginas asignadas a la tarea y seteamos todos los valores iniciales en el struct task correspondiente al Mr. Meeseeks que crearemos.
\\\\
La función use_portal_gun primero checkea que no haya sido llamada por Rick, Morty, ni un Mr. Meeseeks que ya la haya usado. Para elegir un enemigo a teletransportar primero probamos con con número aleatorio y si el enemigo encontrado no era una tarea activa, hacemos una búsqueda lineal hasta encontrar el primer enemigo disponible. Luego elije un par de coordenadas válidas y checkea si hay una semilla en esa posición. Si lo hay, lamentablemente el enemigo absorberá la semilla y será desalojado. Si no hay una semilla en ese lugar, entonces llamamos a _move_meeseeks para transladar al Mr. Meeseeks enemigo a su nueva celda.
\\\\
La función look toma dos punteros a enteros, donde guardará los resultados. Si quien llamó a la función es Rick o Morty, devuelve -1 en ambas variables. En caso contrario, hacemos una búsqueda lineal para encontrar la semilla a menor distancia manhattan del Mr. Meeseeks que llamó al servicio.
